\documentclass[12pt]{article}
\usepackage{MinionPro}
\usepackage{CJK}
%\usepackage[scaled=0.85]{beramono}  %%% scaled=0.775
\usepackage[T1]{fontenc}
\usepackage{graphicx,booktabs,tabularx,psfrag}
\usepackage{xcolor}
\usepackage[a4paper]{geometry}
\usepackage{url}

\usepackage{pstool}

\usepackage{graphicx}
\begin{document}
\begin{CJK}{UTF8}{cwmb}
\renewcommand{\figurename}{圖}

\voffset=-1cm
\textwidth=5.6in
\textheight=9.2in

\newenvironment{num}
 {\leftmargini=6mm\leftmarginii=8mm
  \begin{enumerate}\itemsep=-2pt}
 {\end{enumerate}}

\newenvironment{sol}
 {\begin{quote}\mbox{}\llap{\color{blue}{解答:}\rule{10mm}{0pt}}\hspace*{-4pt}}{\end{quote}}


\thispagestyle{empty}
\fontsize{12}{20pt}\selectfont
\begin{center}
{\large\CJKfamily{cwyb}{經濟學原理下, 習題7}}\\[3mm]
劉彥佑 (R99628130)\\
李卿澄 (B97501046)\\
黃博億 (B99101014)\\
王祉婷 (B00704056)
\end{center}

\begin{num}
\item 
	\begin{num}
		\item 若是保險制度,則類似於完全提撥制,退休者領取的退休金來自於年輕工作時期繳交的「保費」。
		\item 隨收隨付制度下,退休者領取的退休金是來自於當時正在工作的人們所繳交的福利捐。
		\item Support ratio是工作人口對領退休金的人口比例,亦即每個領退休金的人是由多少工作人口來支付其退休金。Support ratio下降將導致工作人口的負擔,能發放的退休金額變小,可能讓工作人口交更多的福利捐或使退休人口領少些退休金,或甚至提高退休年齡。
		\item 政府提高法定退休年齡,在人口年齡結構不改變的狀況下會使工作人口增加,退休人口降低,提高Support ratio。
	\end{num}
\item 
	\begin{num}
		\item 據主計處之國民所得統計常用資料,2010年臺灣國內生產毛額為136,142.21億元,2011年4月底債務餘額占$\frac{48,735}{136,142.21}*100\% = 35.797\%$。
		\item 不能確定。每年都出現財政赤字代表債務餘額還在累積,越來越大,若GDP沒有顯著成長,則債務佔GDP比重將增加;若GDP也有顯著成長,則可能佔比重沒有太大改變,甚至降低。
	\end{num}
\item 
	\begin{num}
		\item
			\begin{num}
				\item 政府儲蓄:稅收-消費支出:$T_1-R_0\bar{B}_0-p_1G_1$
				\item 政府赤字:$\bar{B}_1-\bar{B}_0$
			\end{num} 
		\item 國家財富是政府民間與政府財富之加總。政府赤字擴大,儲蓄減少,政府的財富減少,國家財富亦隨之減少。
	\end{num}
\end{num}

\end{CJK}
\end{document}
