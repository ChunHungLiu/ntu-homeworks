\documentclass[12pt]{article}
\usepackage{MinionPro}
\usepackage{CJK}
%\usepackage[scaled=0.85]{beramono}  %%% scaled=0.775
\usepackage[T1]{fontenc}
\usepackage{graphicx,booktabs,tabularx,psfrag}
\usepackage{xcolor}
\usepackage[a4paper]{geometry}
\usepackage{url}

\usepackage{pstool}

\usepackage{graphicx}
\begin{document}
\begin{CJK}{UTF8}{cwmb}
\renewcommand{\figurename}{圖}

\voffset=-1cm
\textwidth=5.6in
\textheight=9.2in

\newenvironment{num}
 {\leftmargini=6mm\leftmarginii=8mm
  \begin{enumerate}\itemsep=-2pt}
 {\end{enumerate}}

\newenvironment{sol}
 {\begin{quote}\mbox{}\llap{\color{blue}{解答:}\rule{10mm}{0pt}}\hspace*{-4pt}}{\end{quote}}


\thispagestyle{empty}
\fontsize{12}{20pt}\selectfont
\begin{center}
{\large\CJKfamily{cwyb}{經濟學原理下, 習題6}}\\[3mm]
劉彥佑 (R99628130)\\
李卿澄 (B97501046)\\
黃博億 (B99101014)\\
王祉婷 (B00704056)
\end{center}

\begin{num}
\item 
	\begin{num}
		\item $=\frac{102}{2}+\frac{97}{2}=99.5$ 單位。
		\item $=\frac{102}{4}+\frac{100}{2}+\frac{97}{4}=99.75$ 單位。高於(a)小題,因此某甲會購買資產組合。
		\item $=\frac{103}{2}+\frac{97}{2}=100$ 單位。
		\item $=\frac{103}{4}+\frac{100}{2}+\frac{97}{4}=100$ 單位。與(c)小題預期效用相等,故某乙可能購買單一股票亦可能購買資產組合。
	\end{num}
\item 
	\begin{num}
		\item $=(\frac{60+3}{60}-1)*100\%=5\%$。
		\item $=(\frac{58+3}{60}-1)*100\%=1.67\%$。
		\item $=(0.25*(\frac{62+3}{60}-1)+0.5*(\frac{62+3}{60}-1)+0.25*(\frac{58+3}{60}-1))*100\%=5\%$。
	\end{num}
\item 因希臘歐債危機,希臘公債風險上升,相對德國公債風險較低,兩國公債之差額即反映了風險貼水。
\end{num}

\end{CJK}
\end{document}
